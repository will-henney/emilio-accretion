\documentclass[useAMS, usenatbib, a4paper]{mnras}
\pdfsuppresswarningpagegroup=1
\usepackage[spanish,es-minimal,english]{babel}
\usepackage[utf8]{inputenc}
\usepackage{graphicx}
\let\Bbbk\relax
\usepackage{amsmath}	% Advanced maths commands
\usepackage{amssymb}	% Extra maths symbols
\usepackage{xcolor}
\usepackage{fixltx2e}
\usepackage{hyperref}
\usepackage{savesym}
\savesymbol{tablenum}
\usepackage{siunitx}
\restoresymbol{SIX}{tablenum}
\usepackage{newtxtext}
\usepackage[varg,varvw,smallerops]{newtxmath}
\usepackage{xfrac} % for the \sfrac macro
\usepackage{booktabs}
\usepackage{longtable}
\usepackage{array}   % for \newcolumntype macro
\newcolumntype{L}{>{$}l<{$}} % math-mode version of lrc column types
\newcolumntype{R}{>{$}r<{$}} 
\newcolumntype{C}{>{$}c<{$}}
\hypersetup{colorlinks=True, linkcolor=blue!50!black, citecolor=black,
  urlcolor=blue!50!black}
\usepackage{etoolbox}
\robustify\bfseries
\robustify\itshape
\usepackage{enumerate}
\bibliographystyle{mnras}
\sisetup{
  % explicit""+" is useful for velocities
  retain-explicit-plus = true,
  % prefer 10^6 over 1 x 10^6
  retain-unity-mantissa = false,
  % Use x +/- e instead of x(e)  
  separate-uncertainty = true,
  % Make sure to pick up bold font when used in section heading for instance
  detect-weight = true,
  table-align-uncertainty = true,
  table-align-comparator = true,
}
\DeclareSIUnit\msun{\text{M\ensuremath{_\odot}}}
\DeclareSIUnit\lsun{\text{L\ensuremath{_\odot}}}
\DeclareSIUnit\zsun{\text{Z\ensuremath{_\odot}}}
\DeclareSIUnit\angstrom{\text{\AA}}

% A better \ion command that works in more circumstances
\newcommand\ION[2]{#1\,\scalebox{0.9}[0.8]{\uppercase{#2}}}
\newcounter{ionstage}
\renewcommand{\ion}[2]{\setcounter{ionstage}{#2}% 
  \ensuremath{\mathrm{#1\,\scriptstyle\Roman{ionstage}}}}
\newcommand\hii{\ion{H}{2}}



\newcommand\wind{\ensuremath{_{\mathrm{w}}}}
\newcommand\mdwind{\ensuremath{\dot M\wind}}
\newcommand\orb{\ensuremath{_{\mathrm{o}}}}
\newcommand\rel{\ensuremath{_{\mathrm{r}}}}
\newcommand\Hill{\ensuremath{_{\mathrm{\scriptscriptstyle H}}}}
\newcommand\bhl{\ensuremath{_{\mathrm{\scriptscriptstyle B}}}}
\newcommand\Tej{\ensuremath{_{\mathrm{\scriptscriptstyle T}}}}
\newcommand\acc{\ensuremath{_{\mathrm{acc}}}}
\newcommand\mdacc{\ensuremath{\dot M\acc}}

\title[Wind accretion in a circular binary system]
{
  Wind accretion in a circular binary system  
}

\author[Henney et al.]{
  William J. Henney\textsuperscript{1}\thanks{w.henney@irya.unam.mx}
  \\
  \textsuperscript{1}\foreignlanguage{spanish}{%
    Instituto de Radioastronomía y
    Astrofísica, Universidad Nacional Autónoma de México, Apartado
    Postal 3-72, 58090 Morelia, Michoacán, Mexico}\\
}


% These dates will be filled out by the publisher
\date{Accepted XXX. Received YYY; in original form ZZZ}

% Enter the current year, for the copyright statements etc.
\pubyear{2024}


\begin{document}
\label{firstpage}
\pagerange{\pageref{firstpage}--\pageref{lastpage}}
\maketitle



\begin{abstract}
  Commentary on \citet{Tejeda:2025a}.
\end{abstract}

\begin{keywords}
Binary stars -- stellar winds -- stellar accretion
\end{keywords}
%\facilities{VLT:Yepun (MUSE); OANSPM:2.1m (Mezcal); Keck (HIRES)}
%\object{M42}

\section{Introduction}
\label{sec:introduction}

\section{Binary system and wind parameters}
\label{sec:binary-syst-param}

Consider a binary system in which the secondary star with mass \(M_2\) accretes from the wind of the primary star with mass \(M_1\).
The orbit is assumed to be circular with separation \(r\).
The orbital speed of the secondary in the rest frame of the primary is \(v\orb\).
Kepler's laws gives the orbital period \(T\) as
\begin{equation}
  \label{eq:period}
  T = 2\pi G \bigl(M_1 + M_2\bigr) / v\orb^3 = 2\pi r / v_0. 
\end{equation}

The isotropic stellar wind from the primary has mass-loss rate \(\dot M\wind\) and hypersonic terminal velocity \(v\wind\).
The undisturbed wind density \(\rho\wind\) at the position of the secondary is therefore
\begin{equation}
  \label{eq:rho-wind}
  \rho\wind = \mdwind / 4\pi r^2 v\wind. 
\end{equation}

The orbital velocity is purely tangential and thus perpendicular to the purely radial wind velocity. The relative speed between the wind and the secondary star is therefore
\begin{equation}
  \label{eq:relative-velocity}
  v\rel = \Bigl(v\wind^2 + v\orb^2\Bigr)^{1/2}.
\end{equation}

The system is characterized by two dimensionless parameters:
\begin{equation}
  \label{eq:mass-ratio}
  \text{Mass ratio:}\quad q \equiv M_2 / \bigl(M_1 + M_2\bigr),
\end{equation}
\begin{equation}
  \label{eq:velocity-ratio}
  \text{Velocity ratio:}\quad w \equiv v\wind / v\orb.
\end{equation}

\section{Bondi--Hoyle--Littleton accretion}
\label{sec:bondi-hoyle-littl}

Following \citet{Hoyle:1939a, Bondi:1944a},
the mass accretion rate is
\begin{equation}
  \label{eq:mdot-bhl}
  \mdacc = \pi r\bhl^2 \, v\rel \, \rho\wind , 
\end{equation}
where
\begin{equation}
  \label{eq:radius-bhl}
  r\bhl = 2 G M_2 / v\rel^2 
\end{equation}
is the accretion radius.
From equations~(\ref{eq:period}, \ref{eq:relative-velocity}, \ref{eq:mass-ratio}, \ref{eq:velocity-ratio}) we find that the ratio of the accretion radius to the orbital radius is
\begin{equation}
  \label{eq:racc-over-r}
  r\bhl / r = 2 q / \bigl(1 + w^2\bigr).
\end{equation}

We define a dimensionless accretion efficiency as the fraction of the stellar wind that is captured by the secondary:
\begin{equation}
  \label{eq:eta-bhl-def}
  \eta\bhl \equiv \mdacc / \mdwind ,
\end{equation}
which from equations~(\ref{eq:rho-wind}, \ref{eq:mdot-bhl}) yields
\begin{equation}
  \label{eq:eta-bhl}
  \eta\bhl = \frac{1}{4}
  \biggl( \frac{v\rel}{v\wind} \biggr)
  \biggl( \frac{r\bhl}{r}\biggr)^2 . 
\end{equation}
The boost of the relative wind speed due to the orbital motion is
\begin{equation}
  \label{eq:boost-ratio}
  v\rel / v\wind =  w^{-1} \, \bigl(1 + w^2\bigr)^{1/2} ,
\end{equation}
which combine with equations~(\ref{eq:eta-bhl}, \ref{eq:racc-over-r}) implies
\begin{equation}
  \label{eq:eta-bhl-dimensionless}
  \eta\bhl = \frac{q^2}{w \, \bigl(1 + w^2\bigr)^{3/2}} .
\end{equation}

The BHL analysis assumes that the wind density and velocity vector are constant
over the entire accretion capture zone of radius \(r\bhl\).
For accretion from a wind, this is only true in the limit that
\(r\bhl \ll r\). 

\section{A problematic geometric correction}
\label{sec:Tejeda-geometry}

\cite{Tejeda:2025a} point out an issue with equation~(\ref{eq:eta-bhl-dimensionless}) when the wind velocity is much smaller than the orbital velocity (\(w \ll 1\)): in the limit \(w \to 0\) then \(\eta\bhl \to q^2 / w\), which can become larger than unity.
This is clearly non-physical since the mass accretion rate cannot exceed the wind mass-loss rate.
They propose to remedy this deficiency by making a geometric correction
to the mass accretion rate.
\(\mdacc\) is multiplied by a factor of \(\cos\theta\),
where \(\theta\) is the angle between the relative velocity vector
and the radial direction from the primary,
which accounts for
\begin{center}
  \begin{minipage}{0.8\linewidth}\small\itshape
    \dots the projected area of the accretion cylinder's
    cross section onto a sphere centered around the primary. This
    projection accounts for the effective area capturing the wind.
  \end{minipage}
\end{center}
This yields a different equation for the accretion efficiency:
\begin{equation}
  \label{eq:eta-tejeda-dimensionless}
  \eta\Tej = \eta\bhl = \biggl( \frac{q}{1 + w^2} \biggr)^2 .
\end{equation}
This clearly resolves the issue mentioned above since as
\(w \to 0\) then \(\eta\Tej \to q^2 \), guaranteeing that \(\eta\Tej < 1  \).
On the other hand, in the opposite limit of large wind velocity
the two efficiencies agree: as \(w \to \infty\) then \(\eta\Tej \to \eta\bhl \to q^2 / w^4\).

However, the physical basis for making this ``correction'' is unclear.
Unlike the BHL theory, which is entirely local
to the rest frame of the accreting secondary,
the correction factor introduces quantities from the rest frame of the primary,
which casts doubt on its validity.

For circular orbits \(\cos\theta = v\wind / v\rel\), so an alternative
way of writing the Tejeda efficiency is
\begin{equation}
  \label{eq:eta-tejeda}
  \eta\Tej = \frac{1}{4}
  \biggl( \frac{r\bhl}{r}\biggr)^2
  =  \frac{\pi \, r\bhl^2}{4 \pi \, r^2} ,
\end{equation}
which is simply the area covering factor of the BHL capture zone
of a \emph{stationary} accretor.
However, this is inconsistent with the well-defined physical limit
for a fast-orbiting accretor, as we will show in the following section.

\section{Asymptotic efficiency of a fast-orbiting accretor in a slow wind}
\label{sec:fast-orbit-accr}

During one orbital period, \(T\),
the accretion capture zone will sweep out a torus that
fully encircles the primary star
with a total covering factor of \(r\bhl / r\).
If \(T\) is sufficiently short compared with the time \(2 r\bhl / v\wind\) for the wind to cross the accretion capture zone, then all of the wind that passes within
a distance \(r\bhl\) of the orbital path will be captured.
This corresponds to \(w \ll 1\) and yields a limiting accretion efficiency of
\begin{equation}
  \label{eq:1}
  \eta_{\lim} = \lim_{w \to 0} \frac{r\bhl}{r} = 2 q .
\end{equation}
Note that this is inconsistent with the Tejeda result,
\(\eta\Tej \approx q^2\), in the same limit, casting further doubt on
the correctness of that result.

It is also inconsistent with the naive BHL result, \(\eta\bhl \to \infty\),
so we clearly require \emph{some} correction to BHL.
In the following section we outline a physically motivated correction
that is consistent with \(\eta_{\lim}\).

\section{Starvation by finite refill time}
\label{sec:starv-finite-refill}
\cite{Tejeda:2025a} discuss the refill time, which is the time
needed for the wind to replenish the material inside the accretion torus
(see previous section)
but they do not explicitly calculate its influence on the accretion efficiency.
However, we will show that consideration of this effect is
entirely sufficient in order to eliminate
the divergence of \(\eta\bhl\) for small \(w\).

During a single orbital period, the wind will propagate a distance \(v\wind T\),
which represents a fraction \(f\) of the diameter of the accretion torus:
\begin{equation}
  \label{eq:refill-fraction}
  f = \frac{v\wind T}{2 r\bhl} \approx \frac{w (1 + w^2)}{q}
\end{equation}


\section{Gravitational influence of the primary}
\label{sec:limit-hill-sphere}




\bibliography{accretion-refs}



% Don't change these lines
\bsp	% typesetting comment
\label{lastpage}

\end{document}

%%% Local Variables:
%%% mode: LaTeX
%%% TeX-master: t
%%% End:
